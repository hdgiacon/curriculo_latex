\documentclass[
    10pt,
    %A4,
    english,
    %draft = false,
    %twoside = false,
]{article}

\usepackage{curriculum-vitae}

\begin{document}
%	Basic information	
\setname{Héctor Dorrighello}{Giacon}
\setaddress{Maringá / PR}
\setmobile{(+55) 44 99103 6674}
\setmail{hectordorrighello@gmail.com}
\setgithub{github.com/hdgiacon}
\setlinkedin{linkedin.com/in/hectordorrighellodevjunior/}
\setmail{hectordorrighello@gmail.com}

%---------------------------------------------------------------------------------------
%	Title + Contact
%----------------------------------------------------------------------------------------
\cvtitle{Curriculum Vitae}
%---------------------------------------------------------------------------------------
%	Summary / Objectives
%----------------------------------------------------------------------------------------
\cvSection{Objetivos}
\CVTextBlock{Desenvolver atividades técnicas e colaborar com o crescimento da empresa.}
\cvSection{Sumário de Qualificações}
\CVTextBlock{Graduado em Ciência da Computação pela Universidade Estadual de Maringá. Atualmente estou fazendo pós-graduação em Ciência de Dados por meio do HUB de Inteligência Artificial SENAI;}
\\
\noindent \CVTextBlock{No meu tempo livre estudo o framework para desenvolvimento mobile Flutter e conceitos de Clean Code. Tenho experiência com distribuições Linux e versionamento com Git;}
\\
\noindent \CVTextBlock{Em Flutter tenho experiência com clone da UI do instagram, app de comida com gerenciamento de estado em Getx e pagamento por PIX, uma ToDo List com Provider e arquitetura modular;} \\
\noindent \CVTextBlock{Tenho experiência em Python com simulações de robôs, projeto do meu trabalho de conclusão de curso, no qual criei uma biblioteca opensource para localização de robôs móveis;}
\\
\noindent \CVTextBlock{Em Python utilizo em projetos a biblioteca Pandas para exploração de dados, Seaborn para visualização de dados, Streamlit para criação de dashboards e a biblioteca Scikit-learn para modelos de aprendizagem de máquina.}

%---------------------------------------------------------------------------------------
%	Current Position
%----------------------------------------------------------------------------------------
%	\cvSection{Current Position}
%---------------------------------------------------------------------------------------
%	Education
%----------------------------------------------------------------------------------------
\cvSection{Formação Acadêmica}
\CVBlockWithTime{Universidade Estadual de Maringá}{inicio: 04/2016 - término: 06/2022}
{Ciência da Computação}{Maringá, Paraná}{}

%---------------------------------------------------------------------------------------
%	Experience (Research and Industry)
%----------------------------------------------------------------------------------------
%	\cvSection{Experience (Research \& Industry)}
\cvSection{Experiência}
\CVBlockWithTime{HUB de Inteligência Artificial SENAI}{inicio: 09/2022 - término: 09/2023}
{Residência}{Londrina, Paraná}{Compreensão de negócios \& como a inteligência artificial pode ser aplicada à indústria. \\ Casos: \\
    \tab - Problemas de linguagem natural (classificação e similaridade) \\
    \tab - Modelos de visão computacional (GANs e classificação) \\
    \tab - Visualização de dados (Dash)}
%\CVBlockWithTime{Intern}{11/2021 - 2/2022}{Duck GmbH}{That bigger hole, Mars}
%{Implemented additional butchering features into the duck pipeline}
%\CVBlockWithTime{Master Thesis Student}{4 - 10/2021}{Duck GmbH}{That smaller hole, Mars}
%{Evaluation and Adaption of Duck Butchering Algorithms for Mobile Robots in Zero-gravity Environment}

%---------------------------------------------------------------------------------------
%	Projects
%----------------------------------------------------------------------------------------

\cvSection{Projetos Desenvolvidos}
\CVBlockWithTime{I.A. para Linhas de Crédito}{Bunge}{}{2022 - {\small em andamento}}{Utilização de aprendizado de máquina para classificação e regressão para aprovação de linhas de crédito. \\ \textbf{tecnologias:} Pandas, Data Visualization, Random Forest, Flask, REST.}

\newpage

\noindent \CVBlockWithTime{Análise de Tempo em Linha de Produção}{Volvo}{}{2022}{\tab[0.01cm] Análise exploratoria para detecção e emissão de alarmes em pontos de atraso em linha \tab[0.01cm] de produção. \\ \tab[0.01cm] \textbf{tecnologias:} Pandas, Data Visualization, Seaborn, Streamlit.}

\noindent \CVBlockWithTime{I.A. para Horímetros de Maquinas}{Volvo}{}{2023}{\tab[0.01cm] Modelo de regressão para máquinas remotas que não enviam dados regularmente. \\ \tab[0.01cm] \textbf{tecnologias:} Random Forest, LSTM - Tensorflow, Streamlit, Plotly.}

%---------------------------------------------------------------------------------------
%	Skills
%----------------------------------------------------------------------------------------
\cvSection{Competências}
\tab \begin{tabular}{r p{0.7\textwidth}}
    \texttt{\large Linguagens de Programação} & \textbf{Experiência:} Python \cvContactSep Dart \cvContactSep C \cvContactSep Latex \tab \textbf{Familiar:} Javascript/Typescript \cvContactSep SQL \cvContactSep Shell Script                                              \\
    \texttt{\large Frameworks \& Ferramentas} & Pandas \cvContactSep Scikit-learn \cvContactSep Flask \cvContactSep Git \cvContactSep Flutter \cvContactSep  MySQL \cvContactSep Linux \cvContactSep NodeJs \cvContactSep Clean Code \cvContactSep Scrum \cvContactSep REST \\
    \texttt{\large Línguas}                   & \textbf{Intermediário:} Inglês                                                                                                                                                                                              \\
\end{tabular}\\~\\
%---------------------------------------------------------------------------------------
%	Awards and Distinction
%----------------------------------------------------------------------------------------
%\cvSection{Honors \& Awards}
%\CVBlockWithTime{Dean's List}{2021 \& 2022}{Duck University}{}
%{Among the 5 percent best students in the 2019/20 and 2020/21 academic year.}
%---------------------------------------------------------------------------------------
%	Teaching Experience
%----------------------------------------------------------------------------------------
%	\cvSection{Teaching Experience}	
%---------------------------------------------------------------------------------------
%	Extra Curricular Activities
%----------------------------------------------------------------------------------------
%	\newpage
\cvSection{Atividades Extra Curriculares}
\CVBlockWithTime{Academia do Flutter 2.0}{02/2022 - Cursando}{Rodrigo Rahman}{}
{Curso completo sobre Flutter frontend, backend, padrões de projeto, gerenciamento de estado e consumo de API.}

\noindent \CVBlockWithTime{Dart Week 6ª edição}{02/2022}{\tab[0.01cm] Rodrigo Rahman}{}
{\tab[0.01cm] Evento online de 5 dias criando um aplicativo Flutter com integração PIX;}

\noindent \CVBlockWithTime{Dart Week 7ª edição}{06/2022}{\tab[0.01cm] Rodrigo Rahman}{}
{\tab[0.01cm] Evento online de 5 dias criando um aplicativo Flutter com Bloc e Cubit;}

%---------------------------------------------------------------------------------------
%\cvSection{Teaching Experience}
%\cvSection{Professional Activities}
%\cvSection{Invited Talks}
%\cvSection{Selected Press Coverage}
%\cvSection{Publications}
%----------------------------------------------------------------------------------------
\end{document}
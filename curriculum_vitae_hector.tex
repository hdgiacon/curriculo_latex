\documentclass[
    10pt,
    %A4,
    english,
    %draft = false,
    %twoside = false,
]{article}

\usepackage{curriculum-vitae}

\begin{document}
%	Basic information	
\setname{Héctor Dorrighello}{Giacon}
\setaddress{Maringá / PR}
\setmobile{(+55) 44 99103 6647}
\setmail{hectordorrighello@gmail.com}
\setgithub{github.com/hdgiacon}
\setlinkedin{linkedin.com/in/hectordorrighellodev/}
\setmail{hectordorrighello@gmail.com}

%---------------------------------------------------------------------------------------
%	Title + Contact
%----------------------------------------------------------------------------------------
\cvtitle{Curriculum Vitae}
%---------------------------------------------------------------------------------------
%	Summary / Objectives
%----------------------------------------------------------------------------------------
\cvSection{Objetivos}
\CVTextBlock{Busco uma oportunidade como Cientista de Dados onde possa aplicar os meus conhecimentos em análise de dados e inteligência artificial visando extrair o máximo de valor possível do negócio, com novos insights para a companhia como um todo, bem como agregar e colaborar com o time contribuindo com uma parte colaborativa na empresa.}
\cvSection{Sumário de Qualificações}
\CVTextBlock{Bacharel em Ciência da Computação, formado em junho de 2022, constante aprendiz e entusiasmado a conhecer e aprender novas tecnologias para adquirir, reproduzir e compartilhar conhecimento, por meio de iniciativa e trabalho em equipe. Me mudei para Maringá em 2016 para estudar na Universidade Estadual de Maringá e desde então compreendi que tecnologia e aprendizagem juntas podem atender às necessidades dos clientes e das empresas eficientemente.}
\\
\noindent \CVTextBlock{Após a graduação, meu foco tem sido a área de ciência de dados. Fiz uma pós-graduação e residência em ciência de dados pelo HUB de Inteligência Artificial do SENAI Londrina, no qual aprendi conceitos como exploração de dados, machine learning, deep learning, visão computacional, análise e visualização de dados por meio das aulas e projetos reais fornecidos pelas empresas participantes. Dentre elas Volvo, Bunge e Matte Leão.}
\\
\noindent \CVTextBlock{Atualmente estou no SENAI Londrina como bolsista na Fábrica de Software, no qual realizo projetos fornecidos pelas empresas contratantes, aprimorando e adquirindo novos conhecimentos, principalmente na área de séries temporais e com ferramentas como TensorFlow.}
\\
\noindent \CVTextBlock{Paralelamente estou cursando duas matérias do programa de mestrado em informática da Universidade Tecnológica Federal do Paraná de Cornélio Procópio como aluno externo: Inteligência Artificial, no qual estou aprimorando os meus conceitos já aprendidos, com aulas e um trabalho prático entregue no formato de artigo, e Linguagens de Programação, buscando futuramente ingressar no mestrado.}
\\
\noindent \CVTextBlock{Também estou desenvolvendo uma biblioteca Dart/Flutter para modelagem e implementação simplificada de Grafos, e futuramente quero expandir para outras linguagens como Python e Rust. Estudo também conceitos como Padrões de Projeto, Arquiteturas de Software, Clean Code, Rest API e o serviço de análise de negócios e dados Power BI. Tenho experiência em distros Linux como Ubuntu, Manjaro e Microsoft WSL, além da utilização de Git e GitFlow para gerenciamento e versionamento de projetos.}

%---------------------------------------------------------------------------------------
%	Current Position
%----------------------------------------------------------------------------------------
%	\cvSection{Current Position}
%---------------------------------------------------------------------------------------
%	Education
%----------------------------------------------------------------------------------------
\cvSection{Formação Acadêmica}
\CVBlockWithTime{Universidade Estadual de Maringá}{inicio: 04/2016 - término: 06/2022}
{Ciência da Computação - Bacharel}{Maringá, Paraná}{}
\\
\CVBlockWithTime{Hub Inteligência Artificial Senai}{inicio: 09/2022 - término: 09/2023}
{Residência em Inteligência Artificial - Especialização}{Londrina, Paraná}{}
\\
\CVBlockWithTime{Universidade Tecnológica Federal do Paraná}{inicio: 03/2024 - término: 07/2024}
{Aluno externo do mestrado - Inteligência Artificial, Linguagens de Programação}{Cornélio Procópio, Paraná}{}

%---------------------------------------------------------------------------------------
%	Experience (Research and Industry)
%----------------------------------------------------------------------------------------
%	\cvSection{Experience (Research \& Industry)}
\cvSection{Experiência}
\CVBlockWithTime{HUB de Inteligência Artificial SENAI}{inicio: 09/2022 - término: 09/2023}
{Residência}{Londrina, Paraná}{Compreensão de negócios \& como a inteligência artificial pode ser aplicada à indústria. \\ Casos: \\
    \tab - Problemas de linguagem natural (classificação e similaridade) \\
    \tab - Modelos de visão computacional (GANs e classificação) \\
    \tab - Visualização de dados (Dash)}
\\
\CVBlockWithTime{Fábrica de Software SENAI}{inicio: 10/2023 - atualmente}
{Bolsista}{Londrina, Paraná}{Bolsista em Ciência de Dados no qual realizo projetos para as empresas contratantes utilizando estatística e aprendizagem de máquina. \\ \textbf{tecnologias:} Análise Exploratória, Limpeza de Dados, Pandas, Estatística, Séries Temporais, Machine Learning, Deep Learning, TensorFlow.}
%\CVBlockWithTime{Intern}{11/2021 - 2/2022}{Duck GmbH}{That bigger hole, Mars}
%{Implemented additional butchering features into the duck pipeline}
%\CVBlockWithTime{Master Thesis Student}{4 - 10/2021}{Duck GmbH}{That smaller hole, Mars}
%{Evaluation and Adaption of Duck Butchering Algorithms for Mobile Robots in Zero-gravity Environment}

%---------------------------------------------------------------------------------------
%	Projects
%----------------------------------------------------------------------------------------
\cvSection{Projetos Desenvolvidos}
\CVBlockWithTime{Análise Exploratória em Dados para Linhas de Crédito}{Bunge}{}{2022}{Limpeza e análise de dados de clientes a respeito de linhas de crédito. \\ \textbf{tecnologias:} Análise Exploratória, Limpeza de Dados, Pandas, Estatística.}
\\
\CVBlockWithTime{Modelo de Classificação para Linhas de Crédito}{Bunge}{}{2023}{Análise e utilização de inteligência artificial para classificação e aprovação de linhas de crédito por meio de Api. \\ \textbf{tecnologias:} Pandas, Data Visualization, Random Forest - Scikit-learn, Flask, REST.}
\\
\CVBlockWithTime{Modelo de Regressão para Linhas de Crédito}{Bunge}{}{2023}{Utilização de inteligência artificial para regressão em valores de linhas de crédito por meio de Api. \\ \textbf{tecnologias:} Pandas, Data Visualization, LightGBM, Flask, REST.}
\\
\CVBlockWithTime{Automação do Treino de Modelos Mediante aos Dados de Entrada}{Bunge}{}{2023}{Automação e otimização no processo de escolha dos modelos de classificação e regressão mediante aos dados inseridos. \\ \textbf{tecnologias:} Pandas, GridSearch, RandomSearch, Make, Flask, REST.}
\\
\noindent \CVBlockWithTime{Análise de Tempo em Linha de Produção}{Volvo}{}{2022}{Análise exploratoria para detecção e emissão de alarmes em pontos de atraso em linha de produção. \textbf{tecnologias:} Pandas, Data Visualization, Seaborn, Streamlit.}
\\
\noindent \CVBlockWithTime{I.A. para Horímetros de Máquinas}{Volvo}{}{2023}{Modelo de regressão para máquinas remotas que não enviam dados regularmente. \\ \textbf{tecnologias:} Random Forest, LSTM - TensorFlow, Streamlit, Plotly.}
\\
\noindent \CVBlockWithTime{I.A. para Identificação de Padrões de Ervas}{Matte Leão}{}{2023}{Modelo de classificação
    de imagens para identificar ervas conforme os padrões estabelecidos pela empresa. \\ \textbf{tecnologias:} Data Vision, Pandas, MobileNet - PyTorch.}
\\
\noindent \CVBlockWithTime{I.A. para Classificação de Ervas Não Conformes e App Mobile}{Matte Leão}{}{2023}{Otimização do modelo atual, modelo para classificação de não conformidade de ervas e aplicativo mobile para acesso aos modelos de qualquer lugar. \\ \textbf{tecnologias:} Data Vision, Pandas, MobileNet - PyTorch, Flutter.}

%---------------------------------------------------------------------------------------
%	Skills
%----------------------------------------------------------------------------------------
\cvSection{Competências}
\tab \begin{tabular}{r p{0.7\textwidth}}
    \texttt{\large Linguagens de Programação} & \textbf{Experiência:} Python \cvContactSep Dart \cvContactSep C \tab \textbf{Familiar:} Javascript/Typescript \cvContactSep SQL \cvContactSep Shell Script \tab \tab \qquad \qquad \qquad \qquad \qquad \qquad \qquad \quad \cvContactSep Java                                                                                                                                                                                                                                                                                                                                                                                                                                                                                                                  \\
    \texttt{\large Frameworks \& Ferramentas} & Machine Learning                                                                                                                                           \cvContactSep Deep Learning \cvContactSep Análise Exploratória de Dados \cvContactSep Análise de Dados \cvContactSep Visão Computacional \cvContactSep Visualização de Dados \cvContactSep Pandas \cvContactSep Scikit-learn \cvContactSep TensorFlow \cvContactSep Microsoft Power BI \cvContactSep PyTorch \cvContactSep Flask \cvContactSep Git \cvContactSep GitFlow \cvContactSep Make \cvContactSep LaTex \cvContactSep Flutter \cvContactSep  MySQL \cvContactSep Linux \cvContactSep NodeJs \cvContactSep Clean Code \cvContactSep Clean Architecture \cvContactSep Scrum \cvContactSep REST \\
    \texttt{\large Línguas}                   & \textbf{Avançado:} Inglês                                                                                                                                                                                                                                                                                                                                                                                                                                                                                                                                                                                                                                                                                                                                       \\
\end{tabular}\\~\\
%---------------------------------------------------------------------------------------
%	Awards and Distinction
%----------------------------------------------------------------------------------------
%\cvSection{Honors \& Awards}
%\CVBlockWithTime{Dean's List}{2021 \& 2022}{Duck University}{}
%{Among the 5 percent best students in the 2019/20 and 2020/21 academic year.}
%---------------------------------------------------------------------------------------
%	Teaching Experience
%----------------------------------------------------------------------------------------
%	\cvSection{Teaching Experience}	
%---------------------------------------------------------------------------------------
%	Extra Curricular Activities
%----------------------------------------------------------------------------------------
%	\newpage
\cvSection{Atividades Extra Curriculares}
\CVBlockWithTime{Academia do Flutter 2.0}{02/2022 - 12/2023}{Rodrigo Rahman}{}
{Curso online completo sobre Flutter frontend, backend, padrões de projeto, gerenciamento de estado e consumo de API.}
\\
\noindent \CVBlockWithTime{Minicurso de Power BI}{04/2024}{Leonardo Karpinski - Xperiun}{}
{Minicurso introdutório à ferramenta Microsoft Power BI.}

%---------------------------------------------------------------------------------------
%\cvSection{Teaching Experience}
%\cvSection{Professional Activities}
%\cvSection{Invited Talks}
%\cvSection{Selected Press Coverage}
%\cvSection{Publications}
%----------------------------------------------------------------------------------------
\end{document}
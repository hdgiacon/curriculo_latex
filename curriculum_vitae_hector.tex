\documentclass[
    10pt,
    %A4,
    english,
    %draft = false,
    %twoside = false,
]{article}

\usepackage{curriculum-vitae}

\begin{document}
%	Basic information	
\setname{Héctor Dorrighello}{Giacon}
\setaddress{Maringá / PR}
\setmobile{(+55) 44 99103 6647}
\setmail{hectordorrighello@gmail.com}
\setgithub{github.com/hdgiacon}
\setlinkedin{linkedin.com/in/hectordorrighellodev/}
\setmail{hectordorrighello@gmail.com}

%---------------------------------------------------------------------------------------
%	Title + Contact
%----------------------------------------------------------------------------------------
\cvtitle{Curriculum Vitae}
%---------------------------------------------------------------------------------------
%	Summary / Objectives
%----------------------------------------------------------------------------------------
\cvSection{Objetivos}
\CVTextBlock{Busco uma oportunidade como Cientista de Dados onde possa aplicar os meus conhecimentos em análise de dados e inteligência artificial visando extrair o máximo de valor possível do negócio, com novos insights para a companhia como um todo, bem como agregar e colaborar com o time contribuindo com uma parte colaborativa na empresa.}
%---------------------------------------------------------------------------------------
%	Qualifications
%----------------------------------------------------------------------------------------
\cvSection{Sumário de Qualificações}
\CVTextBlock{Sou graduado em Ciência da Computação pela Universidade Estadual de Maringá - UEM (2022) e possuo pós-graduação em Ciência de Dados pelo Hub de Inteligência Artificial do Senai Londrina (2023). Durante minha formação, apliquei técnicas de análise e predição em projetos de classificação e regressão de dados, especialmente em séries temporais e visão computacional.}
\\
\noindent \CVTextBlock{Atuei por um ano como Cientista de Dados na Fábrica de Software do Senai Londrina (2024), onde ampliei meu conhecimento prático em Machine Learning e Deep Learning, experimentando combinações de modelos e técnicas de redes neurais. Também trabalhei com Processamento de Linguagem Natural, estudando e aplicando Large Language Models (LLM) para tarefas que vão desde similaridade de sentenças até geração de textos, além de implementar algoritmos de busca eficientes com estruturas de dados e programação assíncrona.}
\\
\noindent \CVTextBlock{No período noturno, sou mestrando no programa de Mestrado Profissional do PPGI na Universidade Federal Tecnológica do Paraná de Cornélio Procópio - UTFPR (2025). Estou cursando disciplinas como Introdução à Inteligência Artificial, na qual desenvolvi um artigo que compara diferentes modelos de redes neurais para classificação de imagens. Também estudo Mineração de Dados, Processamento de Linguagem Natural (NLP) e Visão Computacional, sendo esta ultima o foco da minha dissertação, no qual aplico Segmanetação Semântica utilizando U-Net e Vision Transformers (ViTs).}
\\
\noindent \CVTextBlock{Atualmente, trabalho como Cientista de Dados na Serrabits, no qual estudo e aplico conhecimentos focados em Visão Computacional e em NLP, com LLMs da arquitetura Transformers.}
\\
\noindent \CVTextBlock{Como hobby, exploro técnicas de visão computacional, como YOLO (You Only Look Once), GANs (Generative Adversarial Networks) e ViT (Vision Transformers). Estou desenvolvendo uma biblioteca Open Source em Dart/Flutter para modelagem e implementação simplificada de grafos, com planos de expandir para outras linguagens, como Python e Rust.}
\\
\\
\\
\noindent \CVTextBlock{Projeto de Visão Computacional:
\\
\url{https://github.com/hdgiacon/cats_and_dogs_classify}}
\\
\noindent \CVTextBlock{Projeto de ChatBot (NLP):
\\
\url{https://github.com/hdgiacon/chat_bot}}

\newpage

%---------------------------------------------------------------------------------------
%	Current Position
%----------------------------------------------------------------------------------------
%	\cvSection{Current Position}
%---------------------------------------------------------------------------------------
%	Education
%----------------------------------------------------------------------------------------
\cvSection{Formação Acadêmica}
\CVBlockWithTime{Universidade Tecnológica Federal do Paraná}{inicio: 02/2025 - término: 12/2026}
{Mestrado em Visão Computacional - Segmentação Semântica}{Cornélio Procópio, Paraná}{}
\\
\CVBlockWithTime{Hub Inteligência Artificial Senai}{inicio: 09/2022 - término: 09/2023}
{Residência em Ciência de dados - Pós-graduação Lato Sensu}{Londrina, Paraná}{}
\\
\CVBlockWithTime{Universidade Estadual de Maringá}{inicio: 04/2016 - término: 06/2022}
{Ciência da Computação - Bacharel}{Maringá, Paraná}{}

%---------------------------------------------------------------------------------------
%	Experience (Research and Industry)
%----------------------------------------------------------------------------------------
%	\cvSection{Experience (Research \& Industry)}
\cvSection{Experiência}
\CVBlockWithTime{Serrabits}{inicio: 11/2024 - Atualmente}
{Cientista de Dados}{Maringá, Paraná \\}{Projetos em Ciência de Dados, nas áreas de: \\
    \tab - análise e visualização de dados com ferramentas de Business Intelligence (BI) \\
    \tab - Visão Computacional \\
    \tab - Processamento de Linguagem Natural com LLMs e modelos generativos}
\\
\CVBlockWithTime{Fábrica de Software SENAI}{inicio: 10/2023 - 10/2024}
{Cientista de Dados}{Londrina, Paraná \\}{Projetos em Ciência de Dados, nas áreas de: \\
    \tab - análise e visualização de dados por meio de dashboards \\
    \tab - comparação de desempenho de modelos clássicos e redes neurais \\
    \tab - séries temporais para previsões mensais e diárias \\
    \tab - Processamento de Linguagem Natural com similaridade e geração de textos - LLM \\
    \tab - otimização de busca em banco de dados com árvores binárias multi dimensionais \\
    \tab - processamento de algoritmos de forma assíncrona a nível de processos}
\\
\CVBlockWithTime{HUB de Inteligência Artificial SENAI}{inicio: 09/2022 - término: 09/2023}
{Residência}{Londrina, Paraná \\}{Compreensão de regras de negócio e como a inteligência artificial pode ser aplicada à indústria, nas áreas de: \\
    \tab - Classificação de imagens com redes neurais em dispositivos moveis \\
    \tab - visualização de dados por meio de dashboards \\
    \tab - modelos de classificação e regressão em dados tabulares \\
    \tab - análise exploratória e estatística sobre dados empresariais}

%---------------------------------------------------------------------------------------
%	Projects
%----------------------------------------------------------------------------------------
\cvSection{Projetos Desenvolvidos}
\CVBlockWithTime{Comparação de desempenho entre modelos convolucionais para o Oxford-IIIT Pet Dataset}{UTFPR}{}{2024}{Projeto de visão computacional para análise e comparação de modelos convolucionais a fim de classificar imagens usando Transfer-learning e GRAD-CAM. \\ \textbf{tecnologias:} Data Vision, Pandas, MobileNet - TensorFlow, EfficientNet - TensorFlow, ResNet - TensorFLow, GRAD-CAM.}

\newpage

\noindent \CVBlockWithTime{Similaridade de sentenças utilizando cosseno, PCA e busca em árvore com KNN}{}{}{2024}{Utilizando a biblioteca Transformers para a criação de vetores de palavras e modelos de redução de dimensionalidade, foi feita a similaridade de sentenças utilizando também estruturas em árvore para otimização no processo de busca. Com o método K Vizinhos Mais Próximos, o tempo de execução foi reduzido de minuto para micro segundos. \\ \textbf{tecnologias:} Processamento de linguagem natural (PLN), K vizinhos mais próximos (KNN), Redução de Dimensionalidade com PCA, Aprendizagem profunda · Django REST Framework, PostgreSQL.}
\\
\CVBlockWithTime{Detecção de Lixo em Ambientes Urbanos com YOLO}{Pessoal}{}{2024}{Projeto de detecção automática de lixo em ambientes urbanos, como ruas e parques. O modelo identifica diferentes residuos, incluindo plástico, metal e papel a partir de imagens. \\ \textbf{tecnologias:} Visão Computacional, YOLO, PyTorch, Ultralytics.}
\\
\CVBlockWithTime{Análise de modelos machine e deep learning para aplicações em séries temporais}{}{}{2024}{Teste e análise de modelos clássicos de Machine Learning, modelos de série temporal baseados no ARIMA e redes neurais para previsão de dados temporais mensais e diários. Modelos de Deep Learning foram testados solo e suas combinações, todos a fim de encontrar aquele que melhor se ajustou à base de dados. \\ \textbf{tecnologias:} Análise de séries temporais, Aprendizado de máquina, Aprendizagem profunda, Aprendizagem profunda · Ciência de dados, Análise de dados.}
\\
\CVBlockWithTime{Fine Tuning de Modelos de I.A. para Classificação de Ervas a App Mobile}{Matte Leão}{}{2023}{Refinamento e testes com novos modelos para aumento de precisão e
acurácia para classificação de ervas Mate e aplicativo mobile para consumo do modelo treinado. \\ \textbf{tecnologias:} Visão Computacional, Pandas, Redes neurais profundas (DNN), PyTorch, Visualização de dados, Flutter.}
\\
\CVBlockWithTime{I.A. para Identificação de Padrões de Ervas}{Matte Leão}{}{2023}{Modelo de classificação
de imagens para identificar ervas conforme os padrões estabelecidos pela empresa. \\ \textbf{tecnologias:} Data Vision, Pandas, MobileNet - PyTorch.}
\\
\CVBlockWithTime{
Inteligência Artificial para Horímetros de Máquinas}{Volvo}{}{2023}{Modelo de regressão para máquinas remotas que não enviam dados regularmente. \\ \textbf{tecnologias:} Random Forest, LSTM - TensorFlow, Streamlit, Plotly.}
\\
\CVBlockWithTime{Modelo de Classificação para Linhas de Crédito}{Bunge}{}{2023}{Análise e utilização de inteligência artificial para classificação e aprovação de linhas de crédito por meio de Api. \\ \textbf{tecnologias:} Pandas, Data Visualization, Random Forest - Scikit-learn, Flask, REST.}
\\
\CVBlockWithTime{Modelo de Regressão para Linhas de Crédito}{Bunge}{}{2023}{Utilização de inteligência artificial para regressão em valores de linhas de crédito por meio de Api. \\ \textbf{tecnologias:} Pandas, Data Visualization, LightGBM, Flask, REST.}
\\
\CVBlockWithTime{Automação do Treino de Modelos Machine Learning}{Bunge}{}{2023}{Automação e otimização no processo de escolha dos modelos de classificação e
regressão mediante aos dados inseridos com GridSearch, RandomSearch, Make e REST API. \\ \textbf{tecnologias:} Pandas, GridSearch, RandomSearch, Make, Flask, REST.}

\newpage

\noindent \CVBlockWithTime{Análise de Tempo em Linha de Produção}{Volvo}{}{2022}{Análise exploratoria para detecção e emissão de alarmes em pontos de atraso em linha de produção. \\ \textbf{tecnologias:} Pandas, Data Visualization, Seaborn, Streamlit.}
\\
\CVBlockWithTime{Análise Exploratória em Dados para Linhas de Crédito}{Bunge}{}{2022}{Limpeza e análise de dados de clientes a respeito de linhas de crédito. \\ \textbf{tecnologias:} Análise Exploratória, Limpeza de Dados, Pandas, Estatística.}
\\
\CVBlockWithTime{Biblioteca Dart para Modelagem Simplificada de Grafos}{Pessoal}{}{atualmente}{Projeto open source para modelagem e implementação simplificada de Grafos para a linguagem Dart. Futuro para Python e Rust. \\ \textbf{tecnologias:} Open Source Library, Dart, Graphs, Object-oriented Programming, Clean Code.}

%---------------------------------------------------------------------------------------
%	Skills
%----------------------------------------------------------------------------------------
\cvSection{Competências}
\tab \begin{tabular}{r p{0.7\textwidth}}
    \texttt{\large Linguagens de Programação} & \textbf{Experiência:} Python \cvContactSep Dart \cvContactSep C \tab \textbf{Familiar:} Javascript/Typescript \cvContactSep SQL \cvContactSep Shell Script \tab \tab \qquad \qquad \qquad \qquad \qquad \qquad \qquad \quad \cvContactSep Java                                                                                                                                                                                                                                                                                                                                                                                                                                                                                                                                                         \\
    \texttt{\large Frameworks \& Ferramentas} & Pandas 
    \cvContactSep Numpy \cvContactSep Scikit-learn \cvContactSep TensorFlow \cvContactSep PyTorch \cvContactSep Transformers \cvContactSep LangChain \cvContactSep Ultralytics \cvContactSep StatsModels \cvContactSep Prophet \cvContactSep FAISS \cvContactSep Django \cvContactSep Celery \cvContactSep Flask \cvContactSep Docker \cvContactSep PostgreSQL \cvContactSep MySQL \cvContactSep Plotly \cvContactSep Microsoft Power BI \cvContactSep REST API \cvContactSep Git \cvContactSep GitFlow \cvContactSep GitHub \cvContactSep GitLab \cvContactSep Make \cvContactSep LaTex \cvContactSep Flutter \cvContactSep Linux \cvContactSep NodeJs \cvContactSep Clean Code \cvContactSep Clean Architecture \cvContactSep Scrum \\
    \texttt{\large Línguas}                   & \textbf{Avançado:} Inglês                                                                                                                                                                                                                                                                                                                                                                                                                                                                                                                                                                                                                                                                                                                                                                              \\
\end{tabular}\\~\\
%---------------------------------------------------------------------------------------
%	Awards and Distinction
%----------------------------------------------------------------------------------------
%\cvSection{Honors \& Awards}
%\CVBlockWithTime{Dean's List}{2021 \& 2022}{Duck University}{}
%{Among the 5 percent best students in the 2019/20 and 2020/21 academic year.}
%---------------------------------------------------------------------------------------
%	Teaching Experience
%----------------------------------------------------------------------------------------
%	\cvSection{Teaching Experience}
%---------------------------------------------------------------------------------------

\cvSection{Publicações}
\CVBlockWithTime{Comparação de desempenho entre modelos convolucionais para o Oxford-IIIT Pet Dataset}{2024}{Sodebras \\}{}
{Artigo escrito sobre o projeto de visão computacional para análise e comparação de modelos convolucionais a fim de classificar imagens usando Transfer-learning e GRAD-CAM. Não publicado na ultima edição ainda, aguardando publicação.}

%---------------------------------------------------------------------------------------
%	Extra Curricular Activities
%----------------------------------------------------------------------------------------
%	\newpage

\cvSection{Atividades Extra Curriculares}
\CVBlockWithTime{Bootcamp Microsoft Copilot AI}{08/2024 - 08/2024}{DIO \\}{}
{Bootcamp focado em apresentar os conceitos que permeiam o Microsoft Copilot, os seus vários tipos de uso, bem como um aprofundamento em Engenharia de Prompt.}
\\
\noindent \CVBlockWithTime{Minicurso de Power BI}{04/2024}{Leonardo Karpinski - Xperiun \\}{}
{Minicurso introdutório à ferramenta Microsoft Power BI.}
\\
\noindent \CVBlockWithTime{Academia do Flutter 2.0}{02/2022 - 12/2023}{Rodrigo Rahman \\}{}
{Curso online completo sobre Flutter frontend, backend, padrões de projeto, gerenciamento de estado e consumo de API.}

%---------------------------------------------------------------------------------------
%\cvSection{Teaching Experience}
%\cvSection{Professional Activities}
%\cvSection{Invited Talks}
%\cvSection{Selected Press Coverage}
%----------------------------------------------------------------------------------------
\end{document}